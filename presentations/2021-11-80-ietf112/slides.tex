\documentclass[aspectratio=169]{beamer}
\usetheme{Boadilla} % plainest one with slide number footer

% generic packages

\usepackage[utf8]{inputenc}
\usepackage[english]{babel}
\usepackage{amsmath}
\usepackage{multicol}
\usepackage{ulem}

% about the presentation
\title{Cacheable OSCORE}
\hypersetup{pdftitle={Cacheable OSCORE}}
\subtitle{
	Or ``What to do when numeric request-response binding fails us''.
	\texttt{draft-amsuess-core-cachable-oscore-03}}
\author{\textit{Christian~Amsüss}, Marco~Tiloca}
\date{2021-11-08, IETF 112}

% used commands

\usepackage{verbatim}

\definecolor{darkgreen}{rgb}{0, 0.56, 0}

% attach self

\usepackage{embedfile}
\embedfile{\jobname.tex}

\begin{document}

\frame{\titlepage}

\begin{frame}{Development since IETF110: It's really two topics}\Large
	\begin{itemize}
		\item[I] How is request-response binding provided -- \\ when the server does not get source authentication?
		\item[II] Once we know that, what do we need for cacheability?
	\end{itemize}

	\bigskip\mbox{}\bigskip

	Split introduced late in -03 -- not as big as feared, but \ldots directions?
\end{frame}

\begin{frame}{Request-response binding in OSCORE}\Large
	What would need to go wrong for response mismatch\footnote{See draft-mattsson-core-coap-attacks-01: CoAP Attacks} to happen?

	\bigskip

	Client intends (and sends) $R1$.

	Server processes (and answers to) $R2$.

	OSCORE ensures sender and seqno match between $R1$ and $R2$.

	\bigskip

	Only client and server can produce such messages, and can thus trust them to be identical.
\end{frame}

\begin{frame}{Request-response binding in Group OSCORE}\Large
	What would need to go wrong for response mismatch to happen?

	\bigskip

	Client $C$ intends (and sends) $R1$.

	Server $S$ processes (and answers to) $R2$.

	OSCORE ensures sender and seqno match\footnote{\ldots and KID context, but that doesn't matter much here} between $R1$ and $R2$.

	\bigskip

	Only $C$ and $S$ can produce such messages \textit{because of source authentication in all messages}.
\end{frame}

\begin{frame}{Who can use a response?}\Large
	In group/group mode, every member can read responses.

	\bigskip

	A third party $T$ can only trust a captured\footnote{Or cached, we'll come to that} response when the original client \textit{and} the server:
	Client $C$ could have sent distinct $R1$ to be seen by $T$, and $R2$ to be seen by $S$.
\end{frame}

\begin{frame}{How can a response be made usable without trusting $C$?}\Large
	\begin{itemize}
		\item Full request is part of response

			e.\,g. a Class E or Class I Response-For\footnote{draft-bormann-core-responses-00: Non-traditional response forms}

		\item Hash of request is part of response (Class I or E)
		\item Either is part of the AAD without being part of the message at all

			e.\,g. by a ``hidden Class I option'' (currently in cacheable), or by extension of external\_aad
	\end{itemize}

	\bigskip

	\ldots replacing / augmenting the (otherwise very practical) request-response binding mechanism.
\end{frame}

\begin{frame}{\ldots and thus, Cacheable OSCORE is split}\Large
	\begin{itemize}
		\item[I] Request-Response binding can be thusly managed -- with some caveats described for Cacheable OSCORE (no freshness)

		\item[II] Deterministic requests become a simple means to create common cache keys, and only deal with avoiding nonce reuse and limited request privacy
	\end{itemize}
\end{frame}


\begin{frame}{Questions}\Large 
	\begin{itemize}
		\item Where else is part I useful?
		\item Is this simpler to follow when presented in split form inside a single document?
	\end{itemize}

	\vspace{2cm}

	Answers? Other questions? Comments?
\end{frame}

\end{document}
