\documentclass[aspectratio=169]{beamer}

% generic packages

\usepackage[utf8]{inputenc}
\usepackage[english]{babel}

% about the presentation
\title{Cachable OSCORE}
\subtitle{\texttt{draft-amsuess-core-cachable-oscore}}
\author{\textit{Christian~Amsüss}, Marco Tiloca}
\date{2020-07-31}

% used commands

\usepackage{verbatim}
\usepackage{soul}

% attach self

\usepackage{embedfile}
\embedfile{\jobname.tex}

\begin{document}

\frame{\titlepage}

\begin{frame}{Background}
	\texttt{multicast-notifications}

	Comparison with ICNs

	OSCON
\end{frame}

\begin{frame}{Caching and OSCORE}\Large
	\begin{eqnarray*}
	\left.
	\begin{array}{l}
		\mbox{POST / 2.01} \\ 
		\mbox{KID and PIV in request}
	\end{array}
	\right\}
	&& \mbox{uncachable}
	\end{eqnarray*}

	\bigskip

	… and it's only one client anyway
\end{frame}

\begin{frame}{For every complex problem, there is a solution…}\Large
	\framesubtitle{that is simple, neat and \st{wrong} insufficient}
	\begin{eqnarray*}
	\left.
	\begin{array}{l}
		\mbox{Group OSCORE} \\
		\mbox{FETCH / 2.05} \\ 
		\mbox{magically hit cache}
	\end{array}
	\right\}
	&& \mbox{verification fails}
	\end{eqnarray*}
\end{frame}

\begin{frame}{Consensus request}\Large
	\begin{itemize}
		\item Pick request sender KID and PIV
		\item Trust in the request\footnote{It'd be a pity if someone requested {\texttt{/whom-i-know}}, and gave you the response claiming they requested {\texttt{/whom-to-trust}}}
	\end{itemize}

	\bigskip

	The ideal candidate to generate a Consensus Request is the server: “Ticket Requests”
\end{frame}

\begin{frame}[fragile]{Ticket Request example}\Large

\begin{verbatim}
Client                Proxy            Server

 enc(GET /a, C:1) ------------------------->
 <--------- enc(Try enc(GET /a, S:1), S/C:1)

 enc(GET /a, S:1) ------>
                    (cache hit)
 <--- enc(2.05 data, S:2)
\end{verbatim}

\bigskip

\footnotesize Assuming pre-existing multicast setup
\end{frame}

\begin{frame}{\texttt{multicast-notifications}'s Phantom Requests are Ticket Requests}\Large
	\begin{enumerate}
		\item Great for observations
		\item Great for large representations\footnote{Unless outer-block mode is used. Which you want. In which case see 3.}
		\item Not so great for everything else
	\end{enumerate}
\end{frame}

\begin{frame}[fragile]{Magically hitting the cache key}\Large

\begin{verbatim}
Client                                 Proxy

 enc(GET /a, C:1), H(/a) ------------------>
 <- enc(2.05 data, S:2) Resp-For enc(GET /a, S:1)
\end{verbatim}

\bigskip

… provided \texttt{H(/a)} is derived the same for every request

\bigskip

(actually it's rather hashing the complete plaintext$|$AAD)

\end{frame}

\begin{frame}[fragile]{Now that we all agree…}\Large
\begin{verbatim}
Client                                 Proxy

 enc(GET /a, C:H(/a)) ------------------->
 <-------------------- enc(2.05 data, S:1)
\end{verbatim}

\pause

\begin{itemize}
	\item Hash over all input to encryption (incl. AAD)
	\item PartIV too short for sufficient hash -- ID-Detail\footnote{Also very nice for B.2 mode}
	\item In group it's encrypt-and-sign -- deterministic client with private key known to group members
\end{itemize}

\end{frame}

\begin{frame}{Questions}\Large
	\begin{itemize}
		\item Practicality
		\item Cryptography
		\item Interest in CoRE
	\end{itemize}
\end{frame}

\end{document}
